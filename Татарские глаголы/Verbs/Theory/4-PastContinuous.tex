\textbf{Критерий использования формы:} действие в прошлом совершается неоднократно или продолжительно.\\ \\
\textbf{Основной глагол} (3л. ед. ч. настоящее время) $+$ \textbf{вспомогательный глагол «иде»} (прошедшее определённое время).\\ \\
Спряжение основного глагола:\\\\
\textbf{-а, -ә} после корня на \underline{согласный.}\\
\textbf{-ый, -и} после корня на \underline{гласный}, причём эта гласная опускается.\\\\
Личные окончания для «иде»:\\\\
\begin{tabular}{ |c|c|c|c| } 
\hline
Число & Лицо & Окончание \\
\hline
\multirow{3}{4em}{ед.}
& I (мин) & \textbf{-м} \\  
& II (син) & \textbf{-ң} \\ 
& III (ул) & - \\ 
\hline
\multirow{3}{4em}{мн.}
& I (без) & \textbf{-к} \\  
& II (сез) & \textbf{-гез} \\ 
& III (алар) & \textbf{-ләр} \\ 
\hline
\end{tabular} \\\\