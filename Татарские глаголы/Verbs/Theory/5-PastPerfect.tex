\textbf{Критерий использования формы:}\\
1) либо действие в прошлом совершено раньше другого действия\\
2) либо действие в прошлом совершено, но не дало желаемый результат.\\\\
\textbf{Основной глагол} (прошедшее неопределённое время) $+$ \textbf{вспомогательный глагол «иде»} (прошедшее определённое время).\\ \\
Спряжение основного глагола:\\\\
\textbf{-кан/-кән} после корня на \underline{глухой согласный.}\\
\textbf{-ган/-гән} после корня в остальных случаях.\\\\
Личные окончания для «иде»:\\\\
\begin{tabular}{ |c|c|c|c| } 
\hline
Число & Лицо & Окончание \\
\hline
\multirow{3}{4em}{ед.}
& I (мин) & \textbf{-м} \\  
& II (син) & \textbf{-ң} \\ 
& III (ул) & - \\ 
\hline
\multirow{3}{4em}{мн.}
& I (без) & \textbf{-к} \\  
& II (сез) & \textbf{-гез} \\ 
& III (алар) & \textbf{-ләр} \\ 
\hline
\end{tabular} \\\\