\textbf{Критерий использования формы:} выражает\\
1) действие, следующее за моментом речи, нейтральное в модальном отношении\\
2) будущее действие с оттенком предположительности\\
3) будущее действие с оттенком категоричности.\\\\
Спрягаемая основа:\\\\
\begin{tabular}{ |c|c|c|c| } 
\hline
Количество слогов & Последний звук & Окончание \\
\hline
\multirow{2}{*}{\multicolumn{1}{c}{больше 1}}
& согласный & \textbf{-ыр/-ер} \\  
& гласный & \textbf{-р} \\ 
\hline
\multirow{3}{*}{\multicolumn{1}{c}{1}}
& р / л & \textbf{-ыр/-ер} \\  
& и / й & \textbf{-яр} \\ 
& всё остальное & \textbf{-ар/-әр} \\ 
\hline
\end{tabular}\\\\
Личные окончания:\\\\
\begin{tabular}{ |c|c|c|c| } 
\hline
Число & Лицо & Окончание \\
\hline
\multirow{3}{4em}{ед.}
& I (мин) & \textbf{-мын/-мен} \\  
& II (син) & \textbf{-сың/-сең} \\ 
& III (ул) & - \\ 
\hline
\multirow{3}{4em}{мн.}
& I (без) & \textbf{-быз/-без} \\  
& II (сез) & \textbf{-сыз/-сез} \\ 
& III (алар) & \textbf{-лар/-ләр} \\ 
\hline
\end{tabular} \\\\\\