\documentclass{article}
\usepackage[utf8]{inputenc}

\usepackage{cmap, todonotes}    
\usepackage{mathtext, indentfirst, amssymb, amsmath, amsthm, mathtools, tabularx}     
\usepackage[T2A]{fontenc}   
\usepackage[utf8]{inputenc}   
\usepackage[english,russian]{babel}
\usepackage[export]{adjustbox}

\title{}
\author{}
\date{}

\begin{document}

{Дополнительное домашнее задание}
\\
\\ Разложить на множители
\\ 1) $x^{3}-1$
\\ 2) $x^{3}+1$
\\ 3) $x^{4}-1$
\\ 4) $x^{3} + x^{2} + x + 1$
\\ 5) $x^{4}+1$
\\ 6) $x^{4} + x^{3} + x^{2} + x + 1$
\\ 7) $x^{6} + x^{3} + 1 = (x^{2}-2\cos{40^\circ}x+1)(x^{2}-2\cos{80^\circ}x+1)(x^{2}-2\cos{160^\circ}x+1)$
\\ 7) $x^{6} + x^{3} + 1 = (x^{2}-2\cos{\frac{2\pi}{9}}x+1)(x^{2}-2\cos{\frac{4\pi}{9}}x+1)(x^{2}-2\cos{\frac{8\pi}{9}}x+1)$
\\
\\

\\ 1) Нарисовать на одной системе координат графики функций 
$y_1=x+4$ и $y_2=-x+2$ и найти их точку пересечения. В ответе точку пересечения написать в виде $(x,y)$
\\ 2) Нарисовать график функции
\begin{equation*}
y= 
 \begin{cases}
   1 &\text{при $x>5$}\\
   -1 &\text{при $x\leq5$}
 \end{cases}
\end{equation*}
\\
\\
\begin{center}1) x^{32} - (x-1)(x+1)(x^2+1)(x^4+1)(x^8+1)(x^{16}+1) =\ ? \end{center} \\
\begin{center}2) x^{32} - (x^2-1)(x^2+1)(x^4+1)(x^8+1)(x^{16}+1) =\ ? \end{center} \\\\
3) Сократить дробь $\frac{x^2-4xy-9+4y^2}{x-2y+3}$
\\4) 60\% от 500 =\ ? \end{center} \\
\\5) 99\% от 500 =\ ? \end{center} \\
\\6) Нарисуй несколько квадратов и закрась в них 50\%, 25\%, 33\%, 66\%, 75\% \\
7) Ученик прочитал 138 страниц, что составляет 23\% числа всех страниц в книге. Сколько страниц в книге? \\

\end{document}