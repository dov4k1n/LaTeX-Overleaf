\documentclass{article}
\usepackage{graphicx} % Required for inserting images
\usepackage[russian]{babel}
\usepackage{hyperref}
\usepackage{amsmath}
\DeclareMathOperator{\tg}{tg}
\newcommand{\tgx}{\tg x}
\usepackage[top=1.5cm, bottom=2cm, left=2.1cm, right=2.1cm]{geometry}
\usepackage{amssymb}
\hypersetup{
    colorlinks=true,
    linkcolor=blue,
    urlcolor=blue,
    pdfborder={0 0 0},
}

\begin{document}

\title{\LARGE{\textbf{Экзаменационные вопросы}}\\[2pt] \large{\textbf{по курсу дифференциальных уравнений}} \\[2pt] \large{\textbf{для 1 потока 2 курса механико-математического факультета}} \\[2pt] \large{\textbf{МГУ им. М.В. Ломоносова в 2022/23 уч. г.}} \\[2pt] \large{\textbf{Лектор – профессор, д.ф.-м.н. Асташова И.В.}}}
\author{\href{https://vk.com/dov4k1n}{Автор файла.} \href{https://drive.google.com/drive/u/0/folders/12TsXEuxM-RLdqzWjB2Olcbk1MkR-lWaz}{Диск с билетами.} \href{https://www.youtube.com/@mm_videoaccelerator}{Ускоренные лекции.}}
\date{18 июня 2023г.}

\maketitle

\section{Осенний семестр}
\begin{enumerate}
    %1
    \item Обыкновенное дифференциальное уравнение, решение, поле направлений, интегральная кривая. Уравнения, разрешённые относительно производной. Связь интегральных кривых и решений уравнения. Построение решений уравнения первого порядка методом изоклин. Метод разделения переменных (формальный).

    %2
    \item Общее решение, частное решение. Задача Коши. Точка единственности решения. Особое решение. Формулировка теоремы существования и единственности решения задачи Коши для дифференциального уравнения первого порядка. Критерий единственности решения задачи Коши для уравнения $y'=f(x)$.

    %3
    \item Примеры особых решений. Обоснование метода разделения переменных. Методы интегрирования однородных уравнений. Уравнения, сводящиеся к однородным. 

    %4
    \item Линейное уравнение 1-ого порядка. Структура общего решения. Методы интегрирования линейных неоднородных уравнений: вариация произвольной постоянной и метод Бернулли. Уравнение Бернулли. Уравнение Риккати.

    %5
    \item Уравнение в дифференциалах и в полных дифференциалах. Необходимое условие уравнения в дифференциалах быть уравнением в полных дифференциалах. Восстановление функции по её полному дифференциалу. Интеграл уравнения. Общий интеграл уравнения.

    %6
    \item Интегрирующий множитель. Пример нахождения интегрирующего множителя. Теорема о существовании интегрирующего множителя. Неединственность интегрирующего множителя. Связь между двумя интегрирующими множителями одного уравнения. Связь интегрирующего множителя и особых решений. Построение общего интеграла по двум известным интегрирующим множителям. Методы нахождения интегрирующего множителя.

    %7
    \item Теорема Пикара о существовании и единственности решения задачи Коши для дифференциальных уравнений первого порядка, разрешённых относительно производной. Лемма об интегральном уравнении.

    %8
    \item Лемма Гронуолла. Следствие из леммы Гронуолла про единственность решения.

    %9
    \item Уравнения первого порядка, не разрешённые относительно производной. Определение продолжения решения на интервал. Дискриминантная кривая. Особое решение. Примеры. Теорема существования и единственности для них. Интегрирование методом введения параметра. Уравнения Лагранжа и Клеро.
    
    %10
    \item Три теоремы о продолжении решения (1. замкнутая ограниченная область. 2. произвольная область и ограниченная функция. 3. произвольная область). Примеры, показывающие, что при выполнении условий локальной теоремы существования и единственности на всей прямой решение уравнения может не продолжаться на всю прямую.

    %11
    \item Лемма о дифференциальном неравенстве. Теорема о продолжении решения на заданный интервал. Пример уравнения, для которого условия теоремы не выполнены, но решения продолжаются.

    %12
    \item Уравнения высокого порядка. Уравнения, разрешённые относительно старшей производной. Задача Коши. Формулировка теоремы Пикара. Линейные уравнения высокого порядка. Формулировка теоремы существования и единственности решения и продолжения решения на весь интервал (в том числе бесконечный) для линейных уравнений. Определение линейной зависимости и независимости функций. Определитель Вронского.

    %13
    \item Необходимое условие линейной зависимости функций. Пример двух функций, когда обратное неверно. Теорема: определитель Вронского линейно независимых функций отличен от нуля. Достаточное условие линейной зависимости функций.

    %14
    \item Формула Лиувилля-Остроградского.

    %15
    \item Определение общего решения уравнения высокого порядка. Свойства решений линейного однородного уравнения высокого порядка. Определение фундаментальной системы решений. Изоморфизм множества решений линейной однородной системы и $\mathbb{R}^{n}$ как линейных пространств. Построение линейного однородного уравнения по заданным $n$ функциям, определитель Вронского которых не обращается в ноль на заданном интервале. Структура общего решения линейного неоднородного уравнения.

    %16
    \item Метод вариации произвольных постоянных для линейных неоднородных уравнений высокого порядка. Общий вид решений линейных однородных уравнений с постоянными действительными коэффициентами. Характеристический многочлен. Теорема об общем виде частного решения линейного неоднородного уравнения с постоянными действительными коэффициентами и правой частью специального вида (без доказательства).
\end{enumerate}

\section{Весенний семестр}
\begin{enumerate}
  %1
  \item Общий вид системы дифференциальных уравнений первого порядка. Нормальная система.

  %2
  \item Формулировка теоремы существования и единственности решения задачи Коши для нормальных систем.

  %3
  \item Линейная однородная, неоднородная, автономная системы.

  %4
  \item Сведение линейной системы n дифференциальных уравнений первого порядка к уравнению n-го порядка.

  %5
  \item Сведение уравнения n-го порядка к линейной системе n дифференциальных уравнений первого порядка.

  %6
  \item Линейная зависимость вектор-функций $\bar{x}_1(t), ..., \bar{x}_n(t)$. Фундаментальная система решений однородной системы.

  %7
  \item Определитель Вронского однородной системы, его свойства.

  %8
  \item Определитель Вронского произвольной системы вектор-функций $\bar{z}_1(t), ..., \bar{z}_n(t)$, его свойства.

  %9
  \item Формула Лиувилля-Остроградского для систем.

  %10
  \item Структура общего решения линейной однородной и неоднородной систем.

  %11
  \item Решение линейной однородной системы с постоянными коэффициентами и правой частью специального вида.

  %12
  \item Метод вариации произвольных постоянных. Определение экспоненты матрицы. Экспонента жордановой клетки. Логарифм матрицы.

  %13
  \item Фазовая плоскость, траектории системы на фазовой плоскости, фазовый портрет системы.

  %14
  \item Свойства траекторий.

  %15
  \item Особые точки системы. Классификация особых точек линейных автономных систем на плоскости.

  %16
  \item Описание поведения траекторий вблизи особых точек с использованием жордановой формы матрицы. Седло.

  %17
  \item Описание поведения траекторий вблизи особых точек с использованием жордановой формы матрицы. Узел.

  %18
  \item Описание поведения траекторий вблизи особых точек с использованием жордановой формы матрицы. Вырожденный узел.
  
  %19
  \item Описание поведения траекторий вблизи особых точек с использованием жордановой формы матрицы. Фокус.
  
  %20
  \item Описание поведения траекторий вблизи особых точек с использованием жордановой формы матрицы. Центр.
  
  %21
  \item Особые точки нелинейных автономных систем.
  
  %22
  \item Определение предельного цикла.

  %23
  \item Уравнение Ньютона.
  
  %24
  \item Определение устойчивости, асимптотической устойчивости и неустойчивости по Ляпунову.

  %25
  \item Устойчивость нулевого решения линейной автономной системы в зависимости от корней характеристического уравнения.

  %26
  \item Области устойчивости нулевого решения линейных систем на плоскости.

  %27
  \item Определение полной производной функции $v(t,\bar{x})$ в силу нормальной системы.
  
  %28
  \item Теорема Ляпунова об устойчивости.

  %29
  \item Теоремы Ляпунова об асимптотической устойчивости.

  %30
  \item Теорема Четаева о неустойчивости.
  
  %31
  \item Определение устойчивости по первому приближению.
  
  %32
  \item Формулировки критерия Рауса-Гурвица и условия Льенара-Шинара.
  
  %33
  \item Колебательный характер решений уравнения $a_0(x)y'' + a_1(x)y' + a_2(x)y = 0$. Его сведение к уравнению вида $y'' + Q(x)y = 0$.
  
  %34
  \item Теоремы об изолированности нулей решений этого уравнения.

  %35
  \item Определение колеблемости решения на интервале и открытом луче.

  %36
  \item Условие неколеблемости всех решений уравнения.

  %37
  \item Теорема Штурма.

  %38
  \item Теорема сравнения.

  %39
  \item Оценка расстояния между последовательными нулями.

  %40
  \item Теорема Кнезера.

  %41
  \item Краевые задачи для линейного уравнения второго порядка. Функция Грина.
\end{enumerate}

\section{Полезные задачи для подготовки}
\begin{enumerate}
    \item Написать частное решение с неопределёнными коэффициентами (значения коэффициентов не находить) $$y''-8y'+17y=e^{4x}(x^2-3x\sin x)$$ (задача 555 в \href{https://drive.google.com/file/d/1vhpR3QoXntqlk_SoChT8c3pDvJQR_b-S/view?usp=drive_link}{Филиппове}) \\

    \item Решить линейную автономную систему уравнений
    $$
        \begin{cases} \dot{x}=2x-y+z, \\ \dot{y}=x+2y-z, \\ \dot{z}=x-y+2z,
        \end{cases}
    $$
    $$(\lambda_1=1,\ \lambda_2=2,\ \lambda_3=3).$$
    (задача 798 в \href{https://drive.google.com/file/d/1vhpR3QoXntqlk_SoChT8c3pDvJQR_b-S/view?usp=drive_link}{Филиппове}) \\

    \item Решить линейную неоднородную систему
    $$
        \begin{cases} \dot{x}=y+2e^{t}, \\ \dot{y}=-x+t^{2}.
        \end{cases}
    $$
    (задача 826 в \href{https://drive.google.com/file/d/1vhpR3QoXntqlk_SoChT8c3pDvJQR_b-S/view?usp=drive_link}{Филиппове}) \\

    \item Решить методом вариации постоянных
    $$
        \begin{cases} \dot{x}=y+\tg^{2}t-1, \\ \dot{y}=-x+\tg t.
        \end{cases}
    $$
    (задача 846 в \href{https://drive.google.com/file/d/1vhpR3QoXntqlk_SoChT8c3pDvJQR_b-S/view?usp=drive_link}{Филиппове}) \\ 
    
    \item Найти и исследовать особые точки линейных и нелинейных систем. Изобразить фазовые портреты. \\ 
    \url{https://phapl.github.io/phapl.ru.html} \\


    \item \href{https://drive.google.com/file/d/1C0imojIb-j87TLhRoWTZHxLw2uVSfrXs/view?usp=sharing}{Ещё 12 задач с экзамена кафедры ТДС}. \\

    \item Задачи на устойчивость в \href{https://drive.google.com/file/d/1vhpR3QoXntqlk_SoChT8c3pDvJQR_b-S/view?usp=drive_link}{Филиппове со страницы 87}. \\

    \item Задачи на дифференцирование по параметру и уравнениям в частных производных первого порядка в \href{https://drive.google.com/file/d/1vhpR3QoXntqlk_SoChT8c3pDvJQR_b-S/view?usp=drive_link}{Филиппове со страницы 109}. \\

    \item Теоретические задачи в \href{https://drive.google.com/file/d/1vhpR3QoXntqlk_SoChT8c3pDvJQR_b-S/view?usp=drive_link}{Филлипове со страницы 128}. \\ 

    \item Ещё теоретические задачи есть в конце каждого раздела \href{https://drive.google.com/file/d/1pRe2EkuqQ4wAvE6CoZ2OiQcv9ChMByza/view?usp=drive_link}{книги Сергеева И.Н.} В этом файле нет содержания, поэтому \href{https://drive.google.com/file/d/1MEFhGgvkHGQpBDJ4iKgh-basEIwyDFPl/view?usp=drive_link}{загрузил его отдельно тут}. Печатную версию можно купить в аргументе за 799р.
\end{enumerate}
\end{document}
